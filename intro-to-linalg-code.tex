\documentclass{article}
\usepackage{graphicx}
\usepackage{amsmath,amssymb}
\usepackage{amsfonts}

\title{An Introduction to Determinants, Eigenvalues, and Eigenvectors}
\author{Vivan Nyati}
\date{June 2024}

\begin{document}

\maketitle

\section*{Determinants}

The determinant is defined to be a function represented by 

\[
\text{$\det(A)$ where $A$ is a square matrix}
\]

\vspace{1mm}

that produces a number in the field of the matrix's entries which satisfies the following properties:

\begin{enumerate}
    \item Doing a line\footnote{Line refers to either a row or a column} addition (also known as line replacement) on \(A\) does not change \(\det(A)\).
    \item Scaling a line of \(A\) by a scalar \(k\) multiplies the determinant by \(k\).
    \item Swapping two lines of a matrix multiplies the determinant by \(-1\).
    \item The determinant of the identity matrix \(I_n\) is equal to 1 (where \(n\) represents the \(n \times n\) dimensions).
\end{enumerate}

Therefore, the elementary matrices that apply these types of operations modify the determinant as follows:

\begin{enumerate}
    \item \(\det(EA) = \det(E)\det(A)\) and \(\det(AE) = \det(A)\det(E)\) if \(E\) is the elementary matrix that does line addition because \(\det(E) = 1\) and line addition doesn't change the determinant of A.
    \item \(\det(EA) = \det(E)\det(A)\) and \(\det(AE) = \det(A)\det(E)\) if \(E\) is the elementary matrix that multiplies a line by a constant k because \(\det(E) = k\) and line scaling by k increases the determinant by a factor of k.
    \item \(\det(EA) = \det(E)\det(A)\) and \(\det(AE) = \det(A)\det(E)\) if \(E\) is the elementary matrix that swaps two lines because \(\det(E) = -1\) and line swapping changes the sign of the determinant (multiplies the determinant by -1).
\end{enumerate}

Following this, the elementary matrices' determinants are modified like such when transposed:

\begin{enumerate}
    \item \(\det(E) = \det(E^T)\) if \(E\) is the elementary matrix that line replaces since the transpose of this elementary matrix changes the line addition from row to column or column to row. Since both row and column replacement act the same way with the determinant, \(\det(E) = \det(E^T) = 1\).
    \item \(\det(E) = \det(E^T)\) if \(E\) is the elementary matrix that multiplies a line by a constant \(k\) since the transpose of this elementary matrix doesn't change it, and when \(E = E^T\), \(\det(E) = \det(E^T) = k\).
    \item \(\det(E) = \det(E^T)\) if \(E\) is the elementary matrix that swaps two lines because swapping two lines changes the sign of the determinant, and the transposition of this elementary matrix changes the line swapping from row to column or column to row. Since both row and column swapping act the same way with the determinant, \(\det(E) = \det(E^T) = -1\).
\end{enumerate}

Since these are the three types of elementary matrices, we find that for $\forall$ elementary matrix $E$ and matrix $A \in \mathbb{R}^{\hspace{0.5mm}n \times n}$, \hspace{-0.5mm}$\det(EA) = \det(E)\det(A)$, $\det(AE) = \det(A)\det(E)$, and $\det(E) = \det(E^T)$.

\vspace{3mm}

There are many useful characteristics of determinants that arise due to their nature:
\vspace{3mm}


1. \textbf{Singleton Determinant Rule}
\[
\text{$\det(A) = A_{1,1} \forall A \in \mathbb{R}^{1 \times 1}$}
\]
\[
\text{because if you scale $I_{1 \times 1}$ by $k$, you get a matrix $A$ $\in \mathbb{R}^{1 \times 1}$\hspace{-0.25mm} with}
\]
\[
\text{an entry equal to the scalar. Based on the definition of a determinant, the}
\]
\[
\text{ determinant would be $k \times \det(I_{1 \times 1})$ $\rightarrow$ $k \times 1$ $\rightarrow$ $k$. Since $k$ would also}
\]
\[
\text{be the single entry of $A$ $\in \mathbb{R}^{1 \times 1}$\hspace{-0.25mm}, $\det(A) = A_{1,1} \forall A \in \mathbb{R}^{1 \times 1}$\hspace{-0.3mm}.}
\]

2. \textbf{Determinant Inverse Rule}
\[
\det(A) \neq 0 \Longleftrightarrow \exists A^{-1}
\]
\[
\text{since } A \text{ can be written as a product of elementary matrices}
\]
\[
A = \prod_{i=1}^{n} E_i \cdot I, \text{ and thus, }
\det(A) = \prod_{i=1}^{n} \det(E_i) \cdot \det(I) \rightarrow \prod_{i=1}^{n} \det(E_i)
\]
\[
\text{where each elementary matrix has a determinant of } 1, -1, \text{ or } k \in \mathbb{R} \cup k \neq 0.
\]
\[
\text{Therefore, $\prod_{i=1}^{n} \det(E_i) \neq 0$, meaning that $\det(A) \neq 0$ if $A^{-1}$ exists.}
\]
\[
\text{If } A^{-1} \text{ doesn't exist, } A \text{ can be row-reduced to a matrix with a zero on the}
\]
\[
\text{diagonal. Hence, $A$ can be expressed as a product of elementary matrices and a}
\]
\[
\text{matrix whose diagonal contains a zero: $\prod_{i=1}^{n} E_i \cdot I'$ where $\det(I') = 0$,}
\]
\[
\text{making $\det(A) = 0$. Thus $\forall A \in \mathbb{R}^{\hspace{0.5mm}n \times n}$\hspace{-0.5mm}, $\det(A) \neq 0 \Longleftrightarrow \exists A^{-1}$}
\]

3. \textbf{Determinant Product Rule}:
\[
\text{$\det(AB) = \det(A) \times \det(B)$}
\]
\[
\text{because for AB to be invertible, $(AB)^{-1}$ must exist such that $AB (AB)^{-1} = I$.}
\]
\[
\text{Through the associative property of matrices we see that $A \left( B (AB)^{-1} \right) = I$,}
\]
\[
\vspace{1mm}
\text{and therefore ‌ $B (AB)^{-1} = A^{-1}$, which would contradict the fact that $A$ is not}
\]
\[
\text{invertible. Therefore, $AB$ is non-invertible only when if $A$ is non-invertible,}
\]
\[
\text{meaning that $\det(AB)  = \det(A)\det(B)$ = 0 when A is non-invertible. When}
\]
\[
\text{$A$ is invertible, it can be written as the product of elementary matrices}
\]
\[
\text{($A = \prod_{k=1}^{n} E_k$,) and since $\det(\prod_{k=1}^{n} E_k \times B)$ expands to $\det(\prod_{k=1}^{n} E_k) \times \det(B)$}
\]
\[
\text{due to the elementary product property, $\det(AB) = \det(A)\det(B)$ when $A$}
\]
\[
\text{is invertible. Therefore, $\forall A, B \in \mathbb{R}^{\hspace{0.5mm}n \times n}$\hspace{-0.5mm}, $\det(AB) = \det(A) \det(B)$.}
\]

4. \textbf{Determinant Transposition Rule}
\[
\text{$\det(A) = \det(A^T)$}
\]
\[
\text{because if $A$ is invertible, $AA^{-1} = A^{-1}A = I_n$, hence,}
\]
\[
\text{$(AA^{-1})^T = (A^{-1}A)^T = I^T$, expanding to $(A^{-1})^TA^T = A^T(A^{-1})^T = I$,}
\]
\[
\text{due to the transpose of a product property. From this, we see that $A^T$ is}
\]
\[
\text{invertible with its inverse being $(A^{-1})^T\hspace{-0.5mm}$. If and only if $A$ is non-invertible is}
\]
\[
\text{$A^T$ non-invertible. So when $A$ is non-invertible, $\det(A) = \det(A^T) = 0$.}
\]
\[
\text{ When $A$ is invertible, it can be written as the product of elementary}
\]
\[
\text{matrices. Therefore, $A^T$ can be written as $(\prod_{k=1}^{n} E_k)^T$. Through the tranpose}
\] 
\[
\text{of a product property, $A^T$ becomes $\prod_{k=1}^{n} (E_{n-k+1})^T$. Since $\det(E^T) = \det(E)$}
\]
\[
\text{$\forall E$, $A^T = \prod_{k=1}^{n} E_{n-k+1}$. Therefore, $\det(A^T) = \det(\prod_{k=1}^{n} E_{n-k+1})$. Through}
\]
\[
\text{the determinant product rule, we see that $\det(A^T) = \prod_{k=1}^{n} \det(E_{n-k+1})$.}
\]
\[
\text{which rearranging becomes $\det(A^T) = \prod_{k=1}^{n} \det(E_k)$. Rearranging again turns}
\]
\[
\text{into $\det(\prod_{k=1}^{n} E_k)$ = $\det(A^T)$ through the determinant product rule. Since}
\]
\[
\text{$\prod_{k=1}^{n} E_k$ = $A$, $\det(A) = \det(A^T)$. As such, $\forall A \in \mathbb{R}^{\hspace{0.5mm}n \times n}$ \hspace{-1.5mm}, $\det(A) = \det(A^T)$.}
\]
\vspace{1mm}

5. \textbf{Determinant Inverse Rule}
\[
\text{$(\det(A))^{-1} = \det(A^{-1})$}
\vspace{1mm}
\]
\[
\text{since $\det(I_n)$ = $\det(AA^{-1})$ = 1 = $\det(A)\det(A^{-1})$ due to determinant}
\]
\[
\text{rproduct rule, and therefore, $\frac{1}{\det(A)}$ = $\det(A^{-1})$, the latter which simplifies}
\]
\[
\text{to $(\det(A))^{-1}$. Thus, $\forall A \in \mathbb{R}^{\hspace{0.5mm}n \times n}$ \hspace{-1.5mm}, where $A^{-1}$ exists, $(\det(A))^{-1} = \det(A^{-1})$.}
\vspace{1mm}
\]

6. \textbf{Determinant Scalar Product Rule}:
\[
\det(kA) = k^n \det(A)
\]
\[
\text{since $n$ represents how many rows would be multiplied and for every row in}
\vspace{2mm}
\]
\[
\text{$A$ multiplied by $k$, the determinant scales by a factor of $k$, the determinant}
\]
\[
\text{scales by $\prod_{i=1}^n k$ $\rightarrow$ $k^n$. Therefore, $\forall A \in \mathbb{R}^{\hspace{0.5mm}n \times n}$ \hspace{-1.5mm}, $\det(kA) = k^n\det(A)$.}
\]

\vspace{2mm}

7. \textbf{Determinant Power Rule}
\[
\text{$\det(A^n) = \det(A)^n$}
\]
\[
\text{because $\det(A^n)$ expands into $\det(\prod_{i=1}^n A)$, and due to the determinant product}
\vspace{1mm}
\]

\[
\text{rule, becomes $\prod_{i=1}^n \det(A)$. This can be rewritten as $(\det(A))^n$. Thus,}
\vspace{2mm}
\]
\[
\text{$\forall A \in \mathbb{R}^{\hspace{0.5mm}n \times n}$ \hspace{-1.5mm}, $(\det(A))^{-1} = \det(A^{-1})$.}
\vspace{4mm}
\]

Now that we've covered some properties, we can compute determinants. One way to do so is through cofactor expansion:

\[
\text{$\det(A) = \sum_{j=1}^{n} (-1)^{i+j} A_{ij} M_{ij}$ where $A \in \mathbb{R}^{\hspace{0.5mm}n \times n}$}
\]

\begin{enumerate}
    \item \(i\) represents the \(i\)th row of A
    \item \(j\) represents the \(j\)th column of A,
    \item \(A_{ij}\) represents the entry of \(A\) at row \(i\), column \(j\), and
    \item \(M_{ij}\) represents the determinant of the submatrix created by removing the \(i\)th row and \(j\)th column of \(A\).
\end{enumerate}

For example, to find
\[
\begin{vmatrix}
4 & 9 & 2 \\
0 & 0 & 3 \\
1 & 5 & 6
\end{vmatrix}
\]

You can use the entries and minors across the second row, producing:
\[
(-1)^{2+1} \cdot 0 \cdot \begin{vmatrix} 9 & 2 \\ 5 & 6 \end{vmatrix} + (-1)^{2+2} \cdot 0 \cdot \begin{vmatrix} 4 & 2 \\ 1 & 6 \end{vmatrix} + (-1)^{2+3} \cdot 3 \cdot \begin{vmatrix} 4 & 9 \\ 1 & 5 \end{vmatrix}
\]

which simplifies to

\[
(-1)^{2+3} \cdot 3 \cdot \begin{vmatrix} 4 & 9 \\ 1 & 5 \end{vmatrix}.
\]

To find

\[
\begin{vmatrix} 4 & 9 \\ 1 & 5 \end{vmatrix}
\]

You can use the entries and minors across the first row:

\[
\begin{vmatrix} 4 & 9 \\ 1 & 5 \end{vmatrix} = (-1)^{1+1} \cdot 4 \cdot \begin{vmatrix} 5 \end{vmatrix} + (-1)^{1+2} \cdot 9 \cdot \begin{vmatrix} 1 \end{vmatrix}
\]

And due to the $1 \times 1$ determinant rule,

\[
\begin{vmatrix} 4 & 9 \\ 1 & 5 \end{vmatrix} = (-1)^{1+1} \cdot 4 \cdot 5 + (-1)^{1+2} \cdot 9 \cdot 1 \rightarrow 11
\]

Plugging that back into the previous equation, we acquire

\[
(-1)^{2+3} \cdot 3 \cdot 11 \rightarrow -33
\]

Therefore, \[
\begin{vmatrix}
4 & 9 & 2 \\
0 & 0 & 3 \\
1 & 5 & 6
\end{vmatrix}
= -33
\]

\section*{Eigenvalues and Eigenvectors}

Both eigenvalues and eigenvectors for a matrix \( A \) are defined by the equation
\[
A x = \lambda x,
\]
where \( \lambda \) is a scalar (representing the eigenvalues) and \( x \) is a vector (representing the eigenvectors). Due to the rules of conformability of matrix multiplication, the resulting dimensions of \( A x \) are the same dimensions as \( x \). Due to scalar multiplication rules, \( \lambda x \) also has the same dimensions as \( x \). This equation can be manipulated:

\begin{enumerate}
    \item \( A x - \lambda x = \mathbf{0} \)
    \item \( x (A - \lambda) = \mathbf{0} \)
    \item \( x (A - \lambda I) = \mathbf{0} \)
\end{enumerate}

Eigenvectors are never zero vectors by convention because, if they were, the eigenvalues could be any element of the complex numbers, rendering both eigenvalues and eigenvectors vacuous. If \( (A - \lambda I)^{-1} \) exists, then we can manipulate the equation further:

\begin{enumerate}
    \item \( x (A - \lambda I)(A - \lambda I)^{-1} = \mathbf{0} (A - \lambda I)^{-1} \)
    \item \( x I = \mathbf{0} \)
    \item \( x = \mathbf{0} \)
\end{enumerate}

Therefore, if \( (A - \lambda I)^{-1} \) exists, \( x = \mathbf{0} \), but since \( x \neq \mathbf{0} \), \( (A - \lambda I)^{-1} \) must not exist. (Note that \( A A^{-1} = I \))

The inverse of a matrix doesn't exist when its determinant is equal to zero. Therefore,
\[
\det(A - \lambda I) = 0.
\]

We've covered how to compute determinants through cofactor expansion, but let's go over it with an arbitrary $A \in \mathbb{R}^{\hspace{0.5mm}n \times n}$

\[
A = \begin{pmatrix} 4 & 9 & 2 \\ 0 & 0 & 3 \\ 1 & 5 & 6 \end{pmatrix}
\]

To find the eigenvalues, compute \( \det(A - \lambda I) = 0 \), where
\[
A - \lambda I = \begin{pmatrix} 4 - \lambda & 9 & 2 \\ 0 & -\lambda & 3 \\ 1 & 5 & 6 - \lambda \end{pmatrix}.
\]

The determinant of \( A - \lambda I \) is:
\[
\det(A - \lambda I) = (4 - \lambda) \begin{vmatrix} -\lambda & 3 \\ 5 & 6 - \lambda \end{vmatrix} - 9 \begin{vmatrix} 0 & 3 \\ 1 & 6 - \lambda \end{vmatrix} + 2 \begin{vmatrix} 0 & -\lambda \\ 1 & 5 \end{vmatrix}.
\]

Expanding these determinants, we get:
\[
\det(A - \lambda I) = (4 - \lambda)(-\lambda (6 - \lambda) - 3 \cdot 5) - 9(0 \cdot (6 - \lambda) - 3 \cdot 1) + 2(0 \cdot 5 - (-\lambda) \cdot 1).
\]

And since $\det(A - \lambda I) = 0$, solving this polynomial for its roots produces our eigenvalue solutions:

\[
\lambda_1 \approx \:-1.42078\dots ,\:\lambda_2\approx \:2.64737\dots ,\:\lambda_3\approx \:8.77341
\]

Now, for each eigenvalue \( \lambda_i \), substitute it back into \( (A - \lambda_i I) x = \vec{0} \) and solve the resulting system of linear equations for the eigenvector \( x \). This is generally done using Gaussian Elimination, but can also be done by the typical computation of systems of equations.

\vspace{1mm}
To do so, we 
\begin{enumerate}
    \item Substitute the respective eigenvalue into the matrix \( A - \lambda I \).
    \item Multiply \( A - \lambda I \) by the variable vector (i.e., \( x, y, z, \dots \)), set the resulting product matrix equal to the zero vector, and represent the matrix equation as linear equations.
    \item Solve the resulting system of equations for each variable.
\end{enumerate}

\vspace{2mm}

Doing this for $A$:

\vspace{4mm}

For $x_1$,

\[
1. \hspace{4mm} A - \lambda_1 I = \begin{pmatrix} 4 + 1.42078 & 9 & 2 \\ 0 & 1.42078 & 3 \\ 1 & 5 & 6 + 1.42078 \end{pmatrix}
\]

\[
\rightarrow \hspace{4mm} A - \lambda_1 I \approx \begin{pmatrix} 5.42078 & 9 & 2 \\ 0 & 1.42078 & 3 \\ 1 & 5 & 7.42078 \end{pmatrix}.
\]

\vspace{2mm}

\[
2. \hspace{4mm} \begin{pmatrix} 5.42078 & 9 & 2 \\ 0 & 1.42078 & 3 \\ 1 & 5 & 7.42078 \end{pmatrix} \begin{pmatrix} x \\ y \\ z \end{pmatrix} = \begin{pmatrix} 0 \\ 0 \\ 0 \end{pmatrix}
\]
\[
\rightarrow \hspace{4mm} \begin{pmatrix} 5.42078 x + 9 y + 2 z \\ 1.42078 y + 3 z \\ x + 5 y + 7.42078 z \end{pmatrix} = \begin{pmatrix} 0 \\ 0 \\ 0 \end{pmatrix}
\]

\[
\hspace{-4mm} \rightarrow \hspace{4mm} 5.42078 x + 9 y + 2 z = 0,
\]
\[
\hspace{-2.5mm} \hspace{-1mm} 1.42078 y + 3 z = 0,
\]
\[
\hspace{-2mm} \hspace{5mm} x + 5 y + 7.42078 z = 0.
\]

\[
\text{\hspace{-4mm} 3. \hspace{2mm} Solving the system of equations, we get $x = 3.11$, $y = 1$, and $z = 2.924.$}
\]

\[
\text{Thus, $x_1 = \begin{pmatrix} 3.11 \\ 1 \\ 2.924 \end{pmatrix}$}
\]
For \( x_2 \),

\[
1. \hspace{4mm} A - \lambda_2 I = \begin{pmatrix} 4 - 2.64737 & 9 & 2 \\ 0 & 2.64737 & 3 \\ 1 & 5 & 6 - 2.64737 \end{pmatrix}
\]

\[
\rightarrow \hspace{4mm} A - \lambda_2 I \approx \begin{pmatrix} 1.35263 & 9 & 2 \\ 0 & 2.64737 & 3 \\ 1 & 5 & 3.35263 \end{pmatrix}.
\]

\vspace{2mm}

\[
2. \hspace{4mm} \begin{pmatrix} 1.35263 & 9 & 2 \\ 0 & 2.64737 & 3 \\ 1 & 5 & 3.35263 \end{pmatrix} \begin{pmatrix} x \\ y \\ z \end{pmatrix} = \begin{pmatrix} 0 \\ 0 \\ 0 \end{pmatrix}
\]
\[
\rightarrow \hspace{4mm} \begin{pmatrix} 1.35263 x + 9 y + 2 z \\ 2.64737 y + 3 z \\ x + 5 y + 3.35263 z \end{pmatrix} = \begin{pmatrix} 0 \\ 0 \\ 0 \end{pmatrix}
\]

\[
\hspace{-4mm} \rightarrow \hspace{4mm} 1.35263 x + 9 y + 2 z = 0,
\]
\[
\hspace{-2.5mm} \hspace{-1mm} 2.64737 y + 3 z = 0,
\]
\[
\hspace{-2mm} \hspace{5mm} x + 5 y + 3.35263 z = 0.
\]

\[
\text{\hspace{-4mm} 3. \hspace{2mm} Solving the system of equations, we get \( x = -6.41 \), \( y = 1 \), and \( z = -0.882 \).}
\]

\[
\text{Thus, \( x_2 = \begin{pmatrix} -6.41 \\ 1 \\ -0.882 \end{pmatrix} \)}
\]

\vspace{10mm}

For \( x_3 \),

\[
1. \hspace{4mm} A - \lambda_3 I = \begin{pmatrix} 4 - 8.77341 & 9 & 2 \\ 0 & 8.77341 & 3 \\ 1 & 5 & 6 - 8.77341 \end{pmatrix}
\]

\[
\rightarrow \hspace{4mm} A - \lambda_3 I \approx \begin{pmatrix} -4.77341 & 9 & 2 \\ 0 & 8.77341 & 3 \\ 1 & 5 & -2.77341 \end{pmatrix}.
\]

\vspace{2mm}

\[
2. \hspace{4mm} \begin{pmatrix} -4.77341 & 9 & 2 \\ 0 & 8.77341 & 3 \\ 1 & 5 & -2.77341 \end{pmatrix} \begin{pmatrix} x \\ y \\ z \end{pmatrix} = \begin{pmatrix} 0 \\ 0 \\ 0 \end{pmatrix}
\]
\[
\rightarrow \hspace{4mm} \begin{pmatrix} -4.77341 x + 9 y + 2 z \\ 8.77341 y + 3 z \\ x + 5 y - 2.77341 z \end{pmatrix} = \begin{pmatrix} 0 \\ 0 \\ 0 \end{pmatrix}
\]

\[
\hspace{-4mm} \rightarrow \hspace{4mm} -4.77341 x + 9 y + 2 z = 0,
\]
\[
\hspace{-2.5mm} \hspace{-1mm} 8.77341 y + 3 z = 0,
\]
\[
\hspace{-2mm} \hspace{5mm} x + 5 y - 2.77341 z = 0.
\]

\[
\text{\hspace{-4mm} 3. \hspace{2mm} Solving the system of equations, we get \( x = -0.822 \), \( y = 1 \), and \( z = -2.924 \).}
\]

\[
\text{Thus, \( x_3 = \begin{pmatrix} -0.822 \\ 1 \\ -2.924 \end{pmatrix} \)}
\]

\end{document}
